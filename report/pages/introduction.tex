\section{Introduction}
\subsection{Introduction option A}.
Increasing concerns for privacy and the wish to protect data against government access, lead to an increasing availability of encryption tools. Nowadays a wide variety of tools is available to everyone with internet access and a basic knowledge of computers.
In respect to Digital Forensics, encryption can pose a big challenge to obtain data from a suspects device. Studies have shown that \gls{FDE} makes it almost impossible to obtain data without getting access to the keys \cite{Casey2011129}. But there are still other ways to access files on a suspects computer. Suspects may make mistakes during their usage of encryption. For example many users use simple passwords containing limited numbers of alphanumerical passwords or dictionary words \cite{worstpractise}, or data is available on non encrypted parts of the disk. Suspects may not do mistakes while using encryption and even hide their encrypted files.
But what if we do not know if there are encrypted files on the computer of the suspect ? Are there ways to find hidden encrypted files on a suspects computer, if we found clues that encryption is used ? Throughout the paper we will look into ways to find Truecrypt containers on a system, how to access them and depict what kind of information can be found about them. We are searching for traces left by Truecrypt on the system and how to use them to access the data which is encrypted in theses containers.
At the end we will compile this data to create a timeline of the suspects actions on the machine and try to find clues what might be hidden inside the encrypted files. 

We begin with a short overview about related literature in section one. We then describe the tool Truecrypt, which we used to create the encrypted data during our experiment in section two. The third section is about the setup, the execution and the outcome of our experiment and in section 4 we describe further techniques to obtain passwords and information about encrypted data, which we not covered during the experiment. We end this paper with our conclusion and describe fields of future work.

\subsection{Introduction option B}
The wish to protect private data against government access increases since the Snowden incident at the latest. 
This also leads to increasing availability of encryption tools and nowadays, a wide variety of tools is available to everyone with a basic knowledge of computers.
\\
In respect to Digital Forensics, encryption can pose a big challenge to obtain data from a suspects device. 
Studies have shown that \gls{FDE} makes it almost impossible to obtain data without getting access to the keys \cite{Casey2011129}. 
But there are other ways to access informations about encrypted data.
Users may make mistakes during their usage of encryption. 
For example many use simple passwords containing limited numbers of alphanumerical passwords or dictionary words \cite{worstpractise}, or data is available on non encrypted parts of the disk. 
But what if the existence of encrypted containers on the suspects hard disks is unknown?
\\
Throughout the paper we will look into ways to identify Truecrypt containers on a system, how to access them and depict what kind of metadata can be retrieved.
We will search for traces, left by Truecrypt on the system and how to use them to access the containers encrypted data.
The paper begins with a short overview about related literature in section one. 
It then describes the Truecrypt software, which is used to create the encrypted data during the experiment in section two. 
The third section describes the setup, execution and the outcome of our experiment. 
Section four describes further techniques to obtain passwords and information about encrypted data, which is not covered during the experiment. 
The paper ends with a conclusion and a describing fields of future work.
