\section{Introduction}
The wish to protect private data against government access increases since the Snowden incident at the latest. 
This also leads to increasing availability of encryption tools and nowadays, a wide variety of tools is available to everyone with a basic knowledge of computers.

In respect to Digital Forensics, encryption can pose a big challenge to obtain data from a suspects device. 
Studies have shown that \gls{FDE} makes it almost impossible to obtain data without getting access to the keys \cite{Casey2011129}. 
But there are other ways to access informations about encrypted data.
Users may make mistakes during their usage of encryption. 
For example many use simple passwords containing limited numbers of alphanumerical passwords or dictionary words \cite{worstpractise}, or data is available on non encrypted parts of the disk. 
But what if the existence of encrypted containers on the suspects hard disks is unknown?

Throughout the paper we will look into ways to identify Truecrypt containers on a system, how to access them and depict what kind of metadata can be retrieved.
We will search for traces, left by Truecrypt on the system and how to use them to access the containers encrypted data.
The paper begins with a short overview about related literature in section one. 
It then describes the Truecrypt software, which is used to create the encrypted data during the experiment in section two. 
The third section describes the setup, execution and the outcome of our experiment. 
Section four describes further techniques to obtain passwords and information about encrypted data, which is not covered during the experiment. 
The paper ends with a conclusion and a describing fields of future work.
