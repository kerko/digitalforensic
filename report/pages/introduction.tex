\section{Introduction}
Increasing concerns for privacy and the wish to protect data against government access, lead to an increasing availability of encryption tools. Nowadays a wide variety of tools is available to everyone with internet access and a basic knowledge of computers.


In respect to Digital Forensics this increases the challenges to obtain data from suspects devices. Studies have shown that \gls{FDE} makes it almost impossible to obtain data without getting access to the keys \cite{casey:11}. But there are still other ways to access files on a suspects computer.  Suspects may make mistakes during their usage of encryption.  For example many users use simple passwords containing limited numbers of alphanumerical passwords or dictionary words \cite{worstpractise} or data is available on non encrypted parts of the disk.

But what can we as forensic investigators do if  the suspect just encrypted some files and did not use \gls{FDE}? Are there techniques available to recover data from the hard drive even when it was encrypted ? Are we able to find left over traces some where on the system?

An overview of the commonly used tools available today will be given, whit explanations what you can achieve with them and how secure the encryption is.
We will try different approaches to recover data which was encrypted on a Windows 7 system.  Recovery of data encrypted by WinRar and Truecrypt will be tried in our experiment, which assumes that \gls{FDE} was not used.
Also we take a look a the common flaws made by suspects during their use of encryption.  How you can exploit these to gain access or find encryption keys.
