\section{Introduction}
Increasing concerns for privacy and the wish to protect data against government access, lead to an increasing availability of encryption tools. Nowadays a wide variety of tools is available to everyone with internet access and a basic knowledge of computers.


In respect to Digital Forensics this increases the challenges to obtain data from suspects devices. Studies have shown that \gls{FDE} makes it almost impossible to obtain data without getting access to the keys \cite{casey:11}. But there are still other ways to access files on a suspects computer.  Suspects may make mistakes during their usage of encryption.  For example many users use simple passwords containing limited numbers of alphanumerical passwords or dictionary words \cite{worstpractice} or data is available on non encrypted parts of the disk. Suspect may not do mistakes while using encryption and even hide their encrypted files.  

But what if we do not know if there are encrypted files on the computer of the suspect ? Are there ways to find hidden encrypted files on a suspects computer, if we found clues that encryption is used ? We will look into ways to find hidden encrypted files, access them and what kind of information can be found about them. We will compile this data to create a timeline of the suspects actions on the machine and try to find clues what might be hidden inside the encrypted files
