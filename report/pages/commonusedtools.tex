\section{True Crypt}
Although the development of the True Crypt software is discontinued since the 28th of May, 2014, it is still one of the most popular encryption software.
It supports encryption for full disks, single files and also containers with the feature of hiding the created containers or partitions.
The available encryption algorithms are AES, Serpent and Twofish. Additionally, there are some combinations possible.
The software runs on Microsoft Windows, Linux, Mac OS X, DragonFly BSD and Android either as a installed or as a portable version.\cite{truecrypt}

A True Crypt container is a fixed size volume which can be created and mounted by the main software.
It contains no file-header nor any detectable sequence for identification.
The name and extension of the container can be arbitrary.

All of these features harden the detectability of a True Crypt container but there are ways to identify them.

Encrypted files contain a high entropy so one approach could be calculating the entropy value for every file in the suspected area and investigate files with high results.
Also checking if the file-header matches the given extension could lead to results.
It is also documented \cite{truecrypt_containerSize} that True Crypt containers always have a size which is dividable by 512 and in addition are always bigger than 292 KB.

The TCHunt tool combines all of these techniques to identify True Crypt containers during its diskscan. It should be mentioned that TCHunt only checks files bigger than 5 MB.
Since all of these file checks are possible with a command line, an own script could be written forr this task.
