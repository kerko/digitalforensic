\section{Further techniques}
There are several other ways to retrieve passwords, data and metadata to
and from encrypted files. Proper encryption is not easy and it can be 
time-consuming, especially if not the whole system is encrypted but 
individual files.\\
\section{Social engineering and retrieving passwords}
One way to retrieve the password to encrypted data is to ask the suspect
to reveal it or to offer a decreased sentence in exchange for the passwords.
This should be considered especially when the data retrieved through decrypting could be used as digital 
evidence in other cases.\\
The simple knowledge, that the suspect has used an encryption program and
perhaps even a program for secure deletion could already be of great use for the
investigators. By emplyoing forensic techniques to retrieve the timestamps 
of these programms, in combination with other information like a timeline
of the incident, an initial suspicion can be hardened. More Information can be
gathered, by looking at the actions and filesused at roughly the same time as the encryption programm.\\
Methods not covered by our experiment also include attacks directed towards 
the password, like Brute-Force, Rainbow-table attacks and dictionary attacks.
These methods aim directly at the password or the password-hash to break it.
To accelerate these methods, it is useful to create a list of potential
password, custom-built to the suspect. While generating this list it is useful 
to include names, birth-dates of family members or pets and
other information retrieved through social networks and the social 
environment of the suspect.
During this reconnaissance it is also useful to extract all strings from all
the confiscated devices and use them as a list. This could be useful since
the suspect might have written and saved the password at some point.
In special cases, where the encrypted data could have a very big impact on the
process of the case, it could be useful to apply dumpster diving. 
That means to look for information in the dumpster of the suspect.
The holy grail of dumpster diving is, to find the password written on a piece
of paper.
Aside that are already created lists with standard and often used passwords
available.
(For example under https://wiki.skullsecurity.org/Passwords)

\section{Retrieving the plaintext}
Often the plaintext, or part of the plaintext can be retrieved from several different places, not only
the encrypted data. This includes intentionall or unintentionall Backups
on internal or external devices and on cloud-systems. 
This could be especially useful on devices where automatic synchronization with
a profile exists, for example to google-cloud.
Other places to look for include temp-folders, thumbnail folders for pictures, and
the history of certain programs used to open the files.

There are several paper describing a few of these techniques in more detail.
\\paper einfügen
