\section{Literature research}

Suspects use different ways to encrypt and hide their data. From \cite{Casey2011129} we learn that, since \gls{FDE} requires additional handling of suspects computers, it is more important than ever to check for encryption before powering of suspects computers. While there are systems to transport running system without ever shutting them of, they make the point that correctly used encryption is not breakable by todays standards.  We also learn from them that the recent changes in law that compel suspects to give up their passwords are no guarantee for access, since suspect can give out wrong passwords and claim them to be the right ones. 

Even if a suspect tries to hide their encrypted volumes using the hidden volume feature of Truecrypt, there are ways to reveal their presence.  They showed in \cite{Hargreaves2010} that when multiple version of the container that houses the hidden volume are available, you can detect changes in the free space of the container which lead to the fact that a hidden volume is present.

We already learned that encrypted disks hinder forensics investigators in their work, but there are ways to gain access to the keys. In \cite{MaartmannMoe2009S132} they had great success extracting keys from memory, the best results are achieved in the states "live system", "screensaver" and "logged out user".

In \cite{5563320} it is shown that encryption tools leave traces on the system, just deleting them does not hide their usage successfully. You can still find registry keys and entries in the prefetch file. All the tools that they inspected left some traces on the system after deletion.

Also there are different ways to obtain data than just the content of the file, in \cite{Castiglione2007750} they showed that metadata of files can be a great source of information. Many users are not aware of the extent of information you can extract from metadata. In some cases metadata even offers the possibility to retrieve delete text. Another paper by \cite{Buchholz2004298} focused on the information a forensics investigator can gather from file system metadata. They come the conclusion that there are short comings and since the metadata can be manipulated, it only helps the investigation to some extent.

